\documentclass[12pt,a4paper]{article}

% Encoding, Czech support
\usepackage[czech]{babel}
\usepackage[utf8]{inputenc}
\usepackage[T1]{fontenc}

% Math + code + images
\usepackage{amsmath}
\usepackage{graphicx}
\usepackage{float}
\usepackage{listings}
\usepackage{caption}
\usepackage{hyperref}
\usepackage{algorithm}
\usepackage{algpseudocode}
\usepackage{booktabs}

% Better Czech quotes
\usepackage{csquotes}

% Page setup
\usepackage{geometry}
\geometry{margin=2.5cm}

\begin{document}

% ============================================================
% Titulní strana ve stylu předchozí práce
% ============================================================
\begin{titlepage}
    \begin{center}
        {\Large Západočeská univerzita v Plzni\\
        Fakulta aplikovaných věd\\
        Katedra informatiky a výpočetní techniky\par}
        
        \vfill
        
        {\LARGE \textbf{Porovnání algoritmů Value Noise, Perlin Noise a Worley Noise}\par}
        \vspace{0.5cm}
        {\large Semestrální práce z předmětu KIV/PRO\par}
        
        \vfill
        
        \begin{flushright}
            \large
            \textbf{Autor:} Vít Novotný\\[0.3em]
            \textbf{Akademický rok:} 2024/2025
        \end{flushright}
    \end{center}
\end{titlepage}

\tableofcontents
\newpage

% ============================================================
\section{Zadání}
% ============================================================

Cílem práce je studium a implementace algoritmů pro procedurální generování dvourozměrného šumu. 
Hlavní metodou je algoritmus \textbf{Perlin noise}, ke kterému jsou pro účely porovnání implementovány algoritmy 
\textbf{Value noise} a \textbf{Worley (cellular) noise}. 

Všechny algoritmy budou implementovány v jazyce Python pod společným rozhraním a porovnávány při stejných parametrech 
(seed, měřítko, rozlišení). Výstupem budou grayscale obrázky, analýza vlastností jednotlivých noise metod a jejich vzájemné porovnání.

% ============================================================
\section{Existující metody}
% ============================================================

V této kapitole jsou popsány tři algoritmy pro generování dvourozměrného šumu. 
U každého je uveden princip, výhody, nevýhody a stručná historie.

\subsection{Value Noise}
Popis principu:
\begin{itemize}
    \item Každému bodu integer mřížky je přiřazena pseudonáhodná hodnota.
    \item Hodnoty mezi body jsou interpolovány (lineárně nebo pomocí funkce fade).
    \item Výsledkem je hladký, ale relativně jednoduchý šum.
\end{itemize}

\subsection{Perlin Noise}
Popis principu:
\begin{itemize}
    \item Gradientový šum navržený Kenem Perlinem (1985).
    \item Každému rohu buňky je přiřazen gradientový vektor.
    \item Výsledná hodnota je dána skalárním součinem a interpolací.
    \item Výhody: hladkost, přirozený vzhled, široké použití v grafice.
\end{itemize}

\subsection{Worley Noise}
Popis principu:
\begin{itemize}
    \item Cellular noise (Steven Worley, 1996).
    \item V každé buňce existuje náhodně umístěný „feature point“.
    \item Hodnota šumu je vzdálenost k nejbližšímu z nich.
    \item Vytváří buněčné a voronoi–podobné struktury.
\end{itemize}

% ============================================================
\section{Zvolené řešení}
% ============================================================

V této kapitole je popsán postup implementace jednotlivých algoritmů 
a sjednocené rozhraní, které umožňuje porovnání metod.

\subsection{Sjednocené rozhraní}

Pro všechny tři algoritmy byla zvolena metoda:
\begin{verbatim}
sample(x : float, y : float) -> float in [0,1]
\end{verbatim}

\subsection{Pseudokód algoritmů}

\subsubsection*{Value Noise – pseudokód}

\begin{algorithm}[H]
\caption{ValueNoise.sample(x, y)}
\begin{algorithmic}
\State $(x_0, y_0) \gets \lfloor x \rfloor, \lfloor y \rfloor$
\State $(x_1, y_1) \gets (x_0 + 1, y_0 + 1)$
\State $v_{00}, v_{10}, v_{01}, v_{11} \gets \text{hashované hodnoty rohů}$
\State $u \gets fade(x - x_0)$
\State $v \gets fade(y - y_0)$
\State \Return bilineární interpolace hodnot pomocí $(u, v)$
\end{algorithmic}
\end{algorithm}

\subsubsection*{Perlin Noise – pseudokód}

\begin{algorithm}[H]
\caption{PerlinNoise.sample(x, y)}
\begin{algorithmic}
\State $(x_0, y_0) \gets \lfloor x \rfloor, \lfloor y \rfloor$
\State $(x_1, y_1) \gets (x_0 + 1, y_0 + 1)$
\State $g_{00}, g_{10}, g_{01}, g_{11} \gets \text{gradienty rohů}$
\State $n_{00}, n_{10}, n_{01}, n_{11} \gets \text{skalární součiny s gradienty}$
\State $sx \gets fade(x - x_0)$
\State $sy \gets fade(y - y_0)$
\State \Return interpolace $n_{00}, n_{10}, n_{01}, n_{11}$ pomocí $(sx, sy)$
\end{algorithmic}
\end{algorithm}

\subsubsection*{Worley Noise – pseudokód}

\begin{algorithm}[H]
\caption{WorleyNoise.sample(x, y)}
\begin{algorithmic}
\State najdi buňku $(i, j)$ obsahující bod $(x, y)$
\State pro každou buňku v okolí $3 \times 3$:
\State \quad spočítej pozici náhodného feature--pointu
\State \quad spočítej vzdálenost od $(x,y)$
\State \Return nejmenší nalezená vzdálenost (normalizovaná)
\end{algorithmic}
\end{algorithm}

% ============================================================
\section{Experimenty a výsledky}
% ============================================================

Experimenty byly provedeny v jazyce Python. 
Všechny tři noise algoritmy byly testovány se stejnými parametry:

\begin{itemize}
    \item seed = 1234
    \item rozlišení 512$\times$512 pixelů
    \item scale = 0{,}02
\end{itemize}

\subsection{Výsledné obrázky}

\begin{figure}[H]
\centering
\includegraphics[width=0.45\linewidth]{value_noise.png}
\caption{Value Noise}
\end{figure}

\begin{figure}[H]
\centering
\includegraphics[width=0.45\linewidth]{perlin_noise.png}
\caption{Perlin Noise}
\end{figure}

\begin{figure}[H]
\centering
\includegraphics[width=0.45\linewidth]{worley_noise.png}
\caption{Worley Noise}
\end{figure}

\subsection{Porovnání}

Sem doplníš slovní komentář:

\begin{itemize}
    \item hladkost,
    \item přítomnost artefaktů,
    \item charakter struktury,
    \item vhodnost pro různé použití (terén, textury atd.).
\end{itemize}

\subsection{Tabulka výsledků}

Zde můžeš mít např. rozptyl nebo průměrnou gradientní změnu:

\begin{table}[H]
\centering
\begin{tabular}{lccc}
\toprule
Metoda & Minimum & Maximum & Průměrná změna gradientu \\
\midrule
Value Noise & ... & ... & ... \\
Perlin Noise & ... & ... & ... \\
Worley Noise & ... & ... & ... \\
\bottomrule
\end{tabular}
\end{table}

% ============================================================
\section{Závěr}
% ============================================================

Zde stručně shrneš:

\begin{itemize}
    \item co bylo implementováno,
    \item jaké rozdíly byly pozorovány,
    \item kdy se která metoda hodí,
    \item co by šlo dále rozšířit (např. fBm, domain warping).
\end{itemize}

% ============================================================
\section{Literatura}
% ============================================================

\begin{enumerate}
    \item Ken Perlin: \textit{An Image Synthesizer}, ACM SIGGRAPH, 1985.
    \item Steven Worley: \textit{A Cellular Texture Basis Function}, SIGGRAPH, 1996.
    \item Další zdroje, které použiješ.
\end{enumerate}

\end{document}
